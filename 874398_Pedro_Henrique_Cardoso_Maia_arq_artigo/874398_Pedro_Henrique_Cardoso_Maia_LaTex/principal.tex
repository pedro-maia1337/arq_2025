%%%%%%%%%%%%%%%%%%%%%%%%%%%%%%%%%%%%%%%%%%%%%%%%%%%%%%%%%%%%%%%%%%%%%%%%%%%%%%%%%%%%%%%%%%%%%%%%%%%%%%%
%%%%%%%%%%%%%% Template de Artigo Adaptado para Trabalho de Diplomação do ICEI %%%%%%%%%%%%%%%%%%%%%%%%
%% codificação UTF-8 - Abntex - Latex -  							     %%
%% Autor:    Fábio Leandro Rodrigues Cordeiro  (fabioleandro@pucminas.br)                            %% 
%% Co-autor: Prof. João Paulo Domingos Silva, Harison da Silva e Anderson Carvalho                   %%
%% Revisores normas NBR (Padrão PUC Minas): Helenice Rego Cunha e Prof. Theldo Cruz                  %%
%% Versão: 1.1     18 de dezembro 2015                     	                                     %%
%%%%%%%%%%%%%%%%%%%%%%%%%%%%%%%%%%%%%%%%%%%%%%%%%%%%%%%%%%%%%%%%%%%%%%%%%%%%%%%%%%%%%%%%%%%%%%%%%%%%%%%


\documentclass[a4paper,12pt,Times]{article}
\usepackage{abakos}  %pacote com padrão da Abakos baseado no padrão da PUC

%%%%%%%%%%%%%%%%%%%%%%%%%%%
%Capa da revista
%%%%%%%%%%%%%%%%%%%%%%%%%%

%\setcounter{page}{80} %iniciar contador de pagina de valor especificado
\newcommand{\monog}{Artigo Científico Arquitetura de Computadores}
\newcommand{\monogES}{Article template Institute of Mathematical Sciences and Informatics}
\newcommand{\tipo}{Artigo }  % Especificar a seção tipo do trabalho: Artigo, Resumo, Tese, Dociê etc
\newcommand{\origem}{Brasil }
\newcommand{\editorial}{Belo Horizonte, p. 01-11, nov. 2015}  % p. xx-xx – páginas inicial-final do artigo
\newcommand{\lcc}{\scriptsize{Licença Creative Commons Attribution-NonCommercial-NoDerivs 3.0 Unported}}

%%%%%%%%%%%%%%%%%INFORMAÇÕES SOBRE AUTOR PRINCIPAL %%%%%%%%%%%%%%%%%%%%%%%%%%%%%%%
\newcommand{\AutorA}{874398 - Pedro Henrique Cardoso Maia}
\newcommand{\funcaoA}{}
\newcommand{\emailA}{pedro.maia.1543726@sga.pucminas.br}
\newcommand{\cursA}{Aluno do Programa de Graduação em Ciência da Computação}

% 
% Definir macros para o nome da Instituição, da Faculdade, etc.
\newcommand{\univ}{Pontifícia Universidade Católica de Minas Gerais}

\newcommand{\keyword}[1]{\textsf{#1}}

\begin{document}
% %%%%%%%%%%%%%%%%%%%%%%%%%%%%%%%%%%
% %% Pagina de titulo
% %%%%%%%%%%%%%%%%%%%%%%%%%%%%%%%%%%

\begin{center}
\includegraphics[scale=0.2]{figuras/brasao.jpg} \\
PONTIFÍCIA UNIVERSIDADE CATÓLICA DE MINAS GERAIS \\
Instituto de Ciências Exatas e de Informática

% \vspace{1.0cm}

\end{center}

 \vspace{0cm} {
 \singlespacing \Large{\monog \symbolfootnote[1]{Artigo apresentado ao Instituto de Ciências Exatas e Informática da Pontifícia Universidade Católica de Minas Gerais como pré-requisito para obtenção do título de Bacharel em Ciência da Computação.} \\ }
  \normalsize{\monogES}
 }

\vspace{1.0cm}

\begin{flushright}
\singlespacing 
\normalsize{\AutorA \footnote{\funcaoA \cursA, \origem -- \emailA . }} \\

\newcommand{\AutorB}{Author Name}

%\normalsize{\AutorC \footnote{\funcaoC \cursC, \origem -- \emailC . }} \\
%\normalsize{\AutorD \footnote{\funcaD \cursD, \origem -- \emailD . }} \\
%deixar com o valor `0` e usar o '*' no inicio da frase
% \symbolfootnote[0]{Artigo recebido em 10 de julho de 1983 e aprovado em 29 de maio 2012}
\end{flushright}
\thispagestyle{empty}

\vspace{1.0cm}

\begin{abstract}
\noindent
Este artigo define e caracteriza tipos de circuitos integrados e suas aplicações com base em artigos científicos e demais fontes.
\\\textbf{\keyword{Palavras-chave: }} Template. \LaTeX. Circuitos integrados, CPLD, FPGA.
\end{abstract}

%%%%%%%%%%%%%%%%%%%%%%%%%%%%%%%%%%%%%%%%%%%%%%%%%%%%%%%%%
 \newpage    %%%% CASO QUEIRA QUE O RESUMO FIQUE EM UMA PAGINA E O ABSTRACT EM OUTRA
\selectlanguage{english}
\begin{abstract}
\noindent
This article defines and characterizes types of integrated circuits and their applications based on scientific articles and other sources.
\\\textbf{\keyword{Keywords: }} Template. \LaTeX. integrated circuits, CPLD, FPGA.
\end{abstract}

\selectlanguage{brazilian}
 \onehalfspace  % espaçamento 1.5 entre linhas
 \setlength{\parindent}{1.25cm}

%%%%%%%%%%%%%%%%%%%%%%%%%%%%%%%%%%%%%%%%%%%%%%%%%
%% INICIO DO TEXTO
%%%%%%%%%%%%%%%%%%%%%%%%%%%%%%%%%%%%%%%%%%%%%%%%%

%%%%%%%%%%%%%%%%%%%%%%%%%%%%%%%%%%%%%%%%%%%%%%%%%%%%%%%%%%%%%%%%%%%%%%%%%%%%%%%%%%%%%%%%%%%%%%%%%%%%%%%
%%%%%%%%%%%%%% Template de Artigo Adaptado para Trabalho de Diplomação do ICEI %%%%%%%%%%%%%%%%%%%%%%%%
%% codificação UTF-8 - Abntex - Latex -  							     %%
%% Autor:    Fábio Leandro Rodrigues Cordeiro  (fabioleandro@pucminas.br)                            %% 
%% Co-autores: Prof. João Paulo Domingos Silva, Harison da Silva e Anderson Carvalho		     %%
%% Revisores normas NBR (Padrão PUC Minas): Helenice Rego Cunha e Prof. Theldo Cruz                  %%
%% Versão: 1.1     18 de dezembro 2015                                                               %%
%%%%%%%%%%%%%%%%%%%%%%%%%%%%%%%%%%%%%%%%%%%%%%%%%%%%%%%%%%%%%%%%%%%%%%%%%%%%%%%%%%%%%%%%%%%%%%%%%%%%%%%

\section{Definição e caracterização} 

   \subsection{ASIC (Application Specific IC)} 
	ASICs é um circuito integrado que é projetado sob medida para aplicações com propósitos específicos. São circuitos implementados para tarefas específicas em funções bem definidas. Essa especialização permite desempenho aprimorado, otimização do uso de espaço no dispositivo porém tem um alto custo. ASICs são amplamente utilizados na eletrônica, de smartphones a dispositivos médicos.

   \subsection{ASSP (Application-Specific Standard Product)} 
   O ASSP também é um circuito integrado que é projetado sob medida para aplicações com propósitos específicos, porém utilizado em funções como codecs de áudio/vídeo, links de comunicação (como USB ou Bluetooth). Ao contrário de um ASICs, um ASSP não é projetado para um único cliente, mas sim um produto padronizado disponibilizado para um mercado amplo.

   \subsection{SPLD (Simple Programmable Logic Devices) } 
   Os SPLDs são formas mais simples e mais baratas de dispositivos lógicos. Eles são utilizados em placas para substituir componentes lógicos (portas AND, OR, NOT). São dispositivos que podem ser apagados e reprogramados por meio de software em vez de alterar o hardware por conta dessa característica, utilizam memórias não voláteis como EPROM, EEPROM.

   \subsection{CPLD (Complex Programmable Logic Device)} 
   Dispositivos Lógicos Programáveis Complexos (CPLDs) são circuitos integrados digitais com células lógicas programáveis e interconexões usadas para implementar uma variedade de funções lógicas digitais para uma ampla gama de aplicações. O princípio básico dos CPLDs é implementar funções lógicas digitais por meio de unidades lógicas programáveis e interconexões programáveis. Essas unidades lógicas podem ser configuradas como uma variedade de portas lógicas e flip-flops, e a rede de interconexão permite que essas unidades lógicas sejam interconectadas para realizar a função de circuito desejada. OS CPLDs utilizam memória não volátil 

   \subsection{SoC (System on a Chip)}
   Um SoC (Sistema em um Chip) é um circuito integrado único que contém todos os componentes essenciais de um sistema eletrônico, como CPU, memória e portas I/O. Fazendo essa integração em um único chip, os dispositivos ficam mais compactos, energeticamente eficientes e rápidos. São bastante utilizados em dispositivos móveis e dispositivos IoT. 

   \subsection{FPGA (Field Programmable Gate Array)} 
   Uma FPGA (matriz de portas programáveis em campo) é um circuito integrado que é projetado para ser configurado pelo usuário, diferentamente de circuitos integrados específicos como os ASICs, esse circuito é programável para atender diferentes propósitos sem a necessidade de modificar ou alterar fisicamente o hardware.

\section{\esp DIFERENCIAÇÃO (PROM, PLA e PAL)}

\begin{table}[ht]
    \centering
    \caption{Diferenças entre PROM, PLA e PAL}
    \begin{tabular}{|m{4.2cm}|m{4.2cm}|m{4.2cm}|}
        \hline
        \textbf{PROM (Memória Somente Leitura Programável)} & \textbf{PLA (Matriz Lógica Programável)} & \textbf{PAL (Lógica de Matriz Programável)} \\
        \hline
        Baseada em memória (armazena dados/instruções). & Usa portas AND e OR. & Usa portas AND e OR. \\
        \hline
        O usuário insere dados, que não podem ser alterados após a inserção. & Ambas as portas AND e OR são programáveis. & A porta AND é programável, mas a porta OR é fixa (não programável). \\
        \hline
        Armazenamento de dados que não podem ser alterados após a programação (memória somente leitura). & Tarefas de decodificação e dados altamente personalizável. & Funções lógicas diversas, oferecendo segurança e confiabilidade. \\
        \hline
        Permite a inserção de dados específicos pelo usuário. & Altamente personalizável, pois ambas as matrizes são programáveis. & Menos flexível que o PLA, pois a matriz OR é fixa. \\
        \hline
        Simples e fácil de usar & Mais cara e mais complexa que PROM e PAL & Simples e fácil de usar \\
        \hline
    \end{tabular}
    \label{tab:diferencas_spls}
\end{table}

   
\section{\esp DIFERENCIAÇÃO (CPLD FPGA)}

\renewcommand{\arraystretch}{2.0} 

\begin{table}[ht]
    \centering
    \caption{Diferenças entre CPLD e FPGA}
    \begin{tabular}{|m{5.5cm}|m{5.5cm}|}
        \hline
        \textbf{CPLD (Complex Programmable Logic Device)} & \textbf{FPGA (Field-Programmable Gate Array)} \\
        \hline
        Arquitetura centralizada. & Arquitetura distribuída. \\
        \hline
        Blocos Lógicos: Macrocélulas com lógica PAL. & Blocos Lógicos: Blocos Lógicos Configuráveis (CLBs). \\
        \hline
        Menos flexível. & Altamente flexível. \\
        \hline
        Menos custoso. & Mais custoso. \\
        \hline
        Design fixo e mais simples para lógica combinacional. & Altamente configuráveis e suportam lógica combinacional e sequencial. \\
        \hline
        Consomem menos energia, custam menos e inicializam instantaneamente. & Exigem mais energia e memória de configuração externa. \\
        \hline
    \end{tabular}
    \label{tab:diferencas_cpld_fpga_adaptada_espacada}
\end{table}

\nocite{artigo01}
\nocite{artigo02}
\nocite{artigo03}
\nocite{video01}
\nocite{video02}
\nocite{video03}
\nocite{video04}
\nocite{video05}
\nocite{video06}
\nocite{site01}
\nocite{site02}
\nocite{site03}
\nocite{site04}
\nocite{site05}
\nocite{site06}



% \subsection{\esp Trabalhos futuros}
% 
% Sugestões de estudos posteriores são ser adicionados subseção deste capítulo de conclusão.


%%%%%%%%%%%%%%%%%%%%%%%%%%%%%%%%%%%
%% FIM DO TEXTO
%%%%%%%%%%%%%%%%%%%%%%%%%%%%%%%%%%%

% \selectlanguage{brazil}
%%%%%%%%%%%%%%%%%%%%%%%%%%%%%%%%%%%
%% Inicio bibliografia
%%%%%%%%%%%%%%%%%%%%%%%%%%%%%%%%%%%

 \newpage
\singlespace{
\renewcommand\refname{REFERÊNCIAS}
\bibliographystyle{abntex2-alf}
\bibliography{bibliografia}

}

\end{document}


